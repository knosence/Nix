% Created 2025-10-17 Fri 22:12
% Intended LaTeX compiler: pdflatex
\documentclass[11pt]{article}
\usepackage[utf8]{inputenc}
\usepackage[T1]{fontenc}
\usepackage{graphicx}
\usepackage{longtable}
\usepackage{wrapfig}
\usepackage{rotating}
\usepackage[normalem]{ulem}
\usepackage{amsmath}
\usepackage{amssymb}
\usepackage{capt-of}
\usepackage{hyperref}
\author{NaDario Seays Sr.}
\date{\today}
\title{}
\hypersetup{
 pdfauthor={NaDario Seays Sr.},
 pdftitle={},
 pdfkeywords={},
 pdfsubject={},
 pdfcreator={Emacs 30.1 (Org mode 9.7.31)}, 
 pdflang={English}}
\begin{document}

\tableofcontents

\section{From Paycheck to Portfolio – My Automated Dividend Engine}
\label{sec:org6ef440b}
\emph{A hands-off system for building real income from everyday paychecks.}
\subsection{What the Paycheck-to-Portfolio Method Is}
\label{sec:org1cbc181}
Imagine every paycheck you earn quietly becoming an employee of its own—one that never complains, never takes weekends off, and keeps sending money back to you.
That’s the core idea behind Shawn’s Paycheck-to-Portfolio system: instead of parking your income in a checking account where it does nothing, you send it straight into investments that start working immediately. Those investments generate dividends, and those dividends eventually pay your bills.

It’s a simple loop: Paycheck → Portfolio → Dividends → Bills, and Margin acts as the flexible bridge that keeps the loop running smoothly.
\subsubsection{The Cash-Flow Loop Explained}
\label{sec:org09566b7}
\begin{enumerate}
\item \texttt{Paycheck Arrives}: goes straight to your investment account (or, in your case, first
\end{enumerate}
into a savings buffer).
\begin{enumerate}
\item \texttt{Investments Work}: your portfolio—made of funds like YMAX, PLTY, AMZY, CLM, and SPYG—starts earning dividends.
\item \texttt{Dividends Pay Bills}: when those dividends arrive, they’re routed back to your savings where bills are paid.
\item \texttt{Margin Fills Gaps}: if a bill comes due before dividends arrive, your margin account steps in automatically to cover the difference.
\end{enumerate}

Think of it as a waterwheel—your paycheck pours in at the top, investments spin the wheel, and dividends flow out the bottom to power your life.
\subsubsection{What Margin Really Is}
\label{sec:org085ea1e}
Margin is money borrowed against your own investments—like a built-in credit line. It automatically fills the gap if your bills come due before your dividends arrive, or when your dividends are still smaller than that bill payment. You repay it when new dividends or paychecks hit.
\subsubsection{Why it’s powerful:}
\label{sec:orgffe7b1a}
\begin{itemize}
\item You never have to sell your investments to pay bills.
\item You keep your portfolio intact and growing.
\item Margin acts like a cushion—not a crutch—as long as it’s used under 25\% of your portfolio value.
\end{itemize}
\subsubsection{Why It Works for Dividend Investors}
\label{sec:org46b05b7}
\begin{itemize}
\item Dividends are predictable income, not hope-based profits.
\item Margin smooths the timing between earning and spending.
\item Your paycheck becomes seed capital for an income-producing engine.
\item Over time, dividends outgrow margin costs and start paying off margin automatically.
\end{itemize}

Goal: eventually, your portfolio’s dividends pay your bills entirely, and your paycheck
goes 100\% into investments.
\subsection{Your \$905 Savings System and Weekly Cash Flow}
\label{sec:orgdc1a51a}

\subsubsection{The \$905 Buffer Concept}
\label{sec:org559a280}
Your \$905 savings balance isn’t random—it’s deliberate. It’s your system’s circuit
breaker. Every bill you pay passes through this savings account. If the balance ever
dips below the required amount for a bill (for example, your \$880 mortgage), that
payment simply won’t go through. This means your account can’t accidentally
overdraft or send partial payments—it quietly holds back until there’s enough to clear
the bill in full.

That \$905 number acts as your emergency float. It ensures there’s always enough
liquidity to cover your largest expense while keeping the rest of your money invested
and working. It’s protection against chaos, but it also makes automation possible.
\subsubsection{The Margin Bridge}
\label{sec:org40b98d4}
Here’s how your automated rules work:
\begin{itemize}
\item If your savings rises \texttt{above \$905}: the excess flows into your margin account. First, it
\end{itemize}
pays off any borrowed margin balance. If there’s still more left, it builds positive margin
credit.

\begin{itemize}
\item If your savings drops \texttt{below \$890}: your margin account steps in. It first draws from
\end{itemize}
any margin credit you’ve built up. If that’s not enough, it temporarily borrows to refill
your savings back to \$905.

This ensures your savings stays steady even when bills, deposits, and dividends
arrive at different times. Margin is the bridge that keeps the loop unbroken.

\texttt{Flow summary}: Paycheck → Savings → Portfolio → Dividends → Savings ↔
Margin.
\subsubsection{Your Weekly Cash Flow (Keep in mind, M1 may not show the auto transfer results but I have never had any issues)}
\label{sec:orgc38facf}
\begin{center}
\begin{tabular}{rllllll}
\hline
Week & Income & Savings (905) & Savings Tranaction & Portfolio Bal (Starting 2000) & Holding Payment & Margin Bal  (Starting 1000)\\
\hline
1 & Paycheck +\$800 & -\$880 (Mortgage) & \$880 + \$25 = \$905 & \$2,000 + \$125 = \$2,125 & YMAX (\$4.73) 1st & -\$880 (TO Savings: Mortgage)\\
 &  & +\$880 (Margin Top Up) & \$880 - \$905 = \$25 &  & PLTY (21.88) 5th & Starting \$1,000 / Ending \$1,880\\
 &  & --- & --- &  & YMAX (4.73)  5th & ---\\
 &  & +800 (Paycheck) & \$800 + \$905 = \$1705 &  &  & +\$800 (FROM Savings: Paycheck)\\
 &  & +800 (To Margin) & \$800 - \$1705 = \$905 &  &  & Starting \$1,880 / Ending \$1,080\\
 &  & --- & --- &  &  & ---\\
 &  & -\$125 (Portfolio) & \$125 - \$905 = \$780 &  &  & -\$125 (TO Savings: Portfolio)\\
 &  & +\$125 (Margin Top Up) & \$125 + \$780 = \$905 &  &  & Starting \$1,080 / Ending \$1,205\\
 &  & --- & --- &  &  & ---\\
 &  & +\$31.34 (Dividend Payout) & \$31.34 + \$905 = \$873.66 &  &  & +\$31.34 (FROM Savings: Dividend)\\
 &  & -31.34 (To Margin) & \$31.34 - \$873.66 = \$905 &  &  & Starting \$1,205 / Ending \$1,173.66\\
\hline
2 & Paycheck +\$800 & +800 (Paycheck) & \$800 + \$905 = \$1705 & \$2,125 + \$125 = \$2,250 & YMAX (4.73) 12th & +\$800 (FROM Savings: Paycheck)\\
 &  & -800 (To Margin) & \$800 - \$1705 = \$905 &  &  & Starting \$1,173.66 / Ending \$373.66\\
 &  & --- & --- &  &  & ---\\
 &  & -\$125 (Porfolio) & \$125 - \$905 = \$780 &  &  & -\$125 (TO Savings: Portfolio)\\
 &  & +\$125 (Margin Top Up) & \$125 + \$780 = \$905 &  &  & Starting \$373.66 / Ending \$498.66\\
 &  & --- & --- &  &  & ---\\
 &  & -\$120 (Georgia Power) & \$120 - \$905 = \$785 &  &  & -\$120 (TO Savings:Georgia Power)\\
 &  & +\$120 (Margin Top Up) & \$120 + \$785 = \$905 &  &  & Starting \$498.66 / Ending \$618.66\\
 &  & --- & --- &  &  & ---\\
 &  & -\$95 (Google Fi) & \$95 - \$905 = \$810 &  &  & -\$95 (TO Savings: Google Fi)\\
 &  & +\$95 (Margin Top Up) & \$95 + \$810 = \$905 &  &  & Starting \$618.66 / Ending \$713.66\\
 &  & --- & --- &  &  & ---\\
 &  & -\$65 (Dekalb Water) & \$65 - \$905 = \$840 &  &  & -\$65 (TO Savings: Dekalb Water)\\
 &  & +\$65 (Margin Top Up) & \$65 + \$840 = \$905 &  &  & Starting \$713.66 / Ending \$778.66\\
 &  & --- & --- &  &  & ---\\
 &  & +\$4.73 (Dividend Payout) & \$4.73 + \$905 = \$909.73 &  &  & +\$4.73 (FROM Savings: Dividend)\\
 &  & -\$4.73 (To Margin) & \$4.73 - \$909.73 = \$905 &  &  & Starting \$778.66 / Ending \$773.93\\
\hline
3 & Paycheck +\$800 & +800 (Paycheck) & \$800 + \$905 = \$1705 & \$2,250 + \$125 = \$2,375 & AMZY (11.34) 19th & +\$800 (FROM Savings: Paycheck)\\
 &  & -800 (To Margin) & \$800 - \$1705 = \$905 &  & YMAX (4.73) 19th & Starting \$773.93 / Ending (+\$26.07)\\
 &  & --- & --- &  &  & ---\\
 &  & -\$125 (Portfolio) & \$125 - \$905 = \$780 &  &  & -\$125 (TO Savings: Portfolio)\\
 &  & +\$125 (Margin Top Up) & \$125 + \$780 = \$905 &  &  & Starting (+\$26.07) / Ending \$98.93\\
 &  & --- & --- &  &  & ---\\
 &  & -\$500 for Insurance & \$500 + \$905 = \$405 &  &  & -\$500 (TO Savings: Insurance)\\
 &  & +\$500 (Margin Top Up) & \$500 + \$405 = \$905 &  &  & Starting \$98.93 / Ending \$598.93\\
 &  & --- & --- &  &  & ---\\
 &  & -\$80 (Gas South) & \$85 + \$905 = \$840 &  &  & -\$85 (TO Savings: Gas South)\\
 &  & +\$80 (Margin Top Up) & \$85 - \$840 = \$905 &  &  & Starting \$598.93 / Ending \$683.93\\
 &  & --- & --- &  &  & ---\\
 &  & +\$16.07 (Dividend Payout) & \$16.07 - \$905 = \$921.07 &  &  & +\$16.07 (FROM Saving: Dividend)\\
 &  & -\$16.07 (To Margin) & \$16.07 + \$921.07 = \$905 &  &  & Starting \$683.93 / Ending \$667.86\\
\hline
4 & Paycheck +\$800 & +800 (Paycheck) & \$800 + \$905 = \$1705 & \$2,375 + \$125 = \$2,500 & SPYG (0.67) 26th & +\$800 (FROM Savings: Paycheck)\\
 &  & -800 (To Margin) & \$800 - \$1705 = \$905 &  &  & Starting \$667.86 / Ending (+\$132.14)\\
 &  & --- & --- &  &  & ---\\
 &  & -\$125 (Portfolio & \$125 - \$905 = \$780 &  &  & -\$125 (TO Savings: Portfolio)\\
 &  & +\$125 (Margin Top Up) & \$125 + \$780 = \$905 &  &  & Starting (+\$132.14) / Ending (+\$7.14)\\
 &  & --- & --- &  &  & ---\\
 &  & -\$20 (Spotify) & \$20 - \$905 = \$885 &  &  & -\$20 (TO Savings: Spotify)\\
 &  & +\$20 (Margin Top Up) & \$20 + \$885 = \$905 &  &  & Starting (+7.14) / Ending \$12.86\\
 &  & --- & --- &  &  & ---\\
 &  & -\$20 (ChatGPT) & \$20 - \$905 = \$885 &  &  & -\$20 (TO Savings: ChatGPT)\\
 &  & +\$20 (Margin Top Up) & \$20 + \$885 = \$905 &  &  & Starting \$12.86 / Ending \$32.86\\
 &  & --- & --- &  &  & ---\\
 &  & -\$14 (TradingView) & \$14 - \$905 = \$891 &  &  & -\$14 (TO Savings: TradingView)\\
 &  & +\$14 (Margin Top Up) & \$14 + \$891 = \$905 &  &  & Starting \$32.86 / Ending \$46.86\\
 &  & --- & --- &  &  & ---\\
 &  & +\$0.67 (Dividend Payout) & \$0.67 + \$905 = \$905.67 &  &  & +\$0.67 (FROM Savings: Dividend)\\
 &  & -\$0.67 (To Margin) & \$0.67 - \$905.67 = 905 &  &  & Starting \$46.86 / Ending \$46.19\\
\hline
5 & Paycheck +\$800 & +800 (Paycheck) & \$800 + \$905 = \$1705 & \$2,500 + \$125 = \$2,625 & YMAX (4.73) 29th & +\$800 (fROM Savings: Paycheck)\\
 &  & -800 (To Margin) & \$800 - \$1705 = \$905 &  & CLM (5.93) 31st & Starting \$46.19 / Ending (+\$753.81)\\
 &  & --- & --- &  &  & ---\\
 &  & -\$125 (Portfolio & \$125 - \$905 = \$780 &  &  & -\$125 (TO Savings: Portfolio)\\
 &  & +\$125 (Margin Top Up) & \$125 + \$780 = \$905 &  &  & Starting (+\$753.81) / Ending (+\$628.81)\\
 &  & --- & --- &  &  & ---\\
 &  & +\$10.66 (Dividend Payout) & \$10.66 + \$905 = \$894.34 &  &  & +\$10.66 (FROM Savings: Dividend)\\
 &  & -\$10.66 (To Margin) & \$10.66 - \$894.34 = \$905 &  &  & Starting (+\$628.81) / Ending (+\$639.47)\\
 &  &  &  &  &  & \\
 &  &  &  &  &  & \\
 &  &  &  &  &  & \\
 &  &  &  &  &  & \\
 &  &  &  &  &  & \\
 &  &  &  &  &  & \\
 &  &  &  &  &  & \\
 &  &  &  &  &  & \\
 &  &  &  &  &  & \\
 &  &  &  &  &  & \\
\hline
\end{tabular}
\end{center}

\begin{center}
\begin{tabular}{lll}
Account & Starting Amounts & Ending Amounts\\
\hline
Savings & \$905 & \$905\\
Portfolio & \$2,000 & \$2,625\\
Margin & \$1000 & +\$639.47\\
\end{tabular}
\end{center}
\begin{enumerate}
\item Income
\label{sec:orgf1d45e5}
\begin{itemize}
\item Paycheck 3,200
\item Dividend 63.47
\end{itemize}
\item Expenses
\label{sec:org60a375b}
\begin{itemize}
\item 1,794
\end{itemize}
\end{enumerate}
\subsubsection{Automation in Motion}
\label{sec:orgd043ba1}
Your savings account acts as the control tower of the entire system.
\begin{itemize}
\item Every Friday, your paycheck arrives automatically.
\item Immediately, \$125 moves into your M1 portfolio.
\item Bills are paid straight from your savings account.
\item Dividends flow back in, refilling savings.
\end{itemize}

Whenever your savings creeps above \$905, margin gets repaid automatically. When it slips below \$890, margin quietly advances funds to keep payments smooth. It’s a hands-off feedback loop that runs 24/7—no manual transfers, no guesswork, no stress.

Over time, as dividends increase, this same automation begins to repay margin completely—eventually leaving you with a self-sustaining cycle where your money pays your bills and your paycheck simply grows your portfolio.
\subsubsection{Portfolio Growth (for your chart)}
\label{sec:orgb04d92a}
Starting: \$2,000
Add: \$125 weekly (\$500 monthly)
→ Formula:
Portfolio[n] = Portfolio[n-1] + 500

So:

\begin{center}
\begin{tabular}{ll}
Month & Portfolio Value\\
----- & ---------------\\
Jan & \$2,000\\
Feb & \$2,500\\
Mar & \$3,000\\
Apr & \$3,500\\
May & \$4,000\\
Jun & \$4,500\\
Jul & \$5,000\\
Aug & \$5,500\\
Sep & \$6,000\\
Oct & \$6,500\\
Nov & \$7,000\\
Dec & \$7,500\\
\end{tabular}
\end{center}
\subsection{Scaling Phases, Glossary \& Checklist}
\label{sec:org6ba5391}

\subsubsection{Section 1 – Scaling Phases: From \$500 to \$3 200 per Month}
\label{sec:org40fcee3}
Phase	Monthly Portfolio Contribution	Goal	Purpose
Phase 1	\$500	Cover insurance (\textasciitilde{}\$500)	Start cash-flow cycle and build dividend base.
Phase 2	\$1 000	Cover insurance + utilities (\textasciitilde{}\$900–\$1 000)	Reduce margin usage and add weekly dividend stability.
Phase 3	\$1 500	Add mortgage support (\textasciitilde{}\$1 800 bills total)	Portfolio dividends begin to match major expenses.
Phase 4	\$2 000	Replace all bills (\textasciitilde{}\$2 200–\$2 500)	Margin usage minimal; system becomes self-feeding.
Phase 5	\$3 200	Full paycheck → portfolio	Portfolio income > bills; dividends pay life completely.

How to move between phases:
Increase only when average dividends cover the previous phase’s bills for two straight months.
\subsubsection{Section 2 – Key Concept Glossary (18 terms)}
\label{sec:org0aaa5b9}
\begin{center}
\begin{tabular}{ll}
Term & Definition \& Example\\
Portfolio & Your collection of income-producing funds (AMZY, PLTY, YMAX, CLM, SPYG)\\
Dividend & Cash payment from a fund or stock. E.g., \$100 in dividends hits your savings each month.\\
Yield \% & Annual dividends ÷ portfolio value. 33 \% yield on \$6 000 ≈ \$165 per month.\\
Margin Account & Brokerage feature that lets you borrow against investments. Acts as temporary cash bridge.\\
Margin Interest & Fee for using borrowed margin (≈ 7 \% per year).\\
Savings Buffer & The \$905 held to pay bills and avoid overdrafts.\\
Floor \& Ceiling Rule & Below \$890 → borrow from margin. Above \$905 → repay margin.\\
Cash-Flow Loop & Paycheck → Savings → Portfolio → Dividends → Savings ↔ Margin.\\
Automation & Your scheduled transfers that make the loop run without manual input.\\
DRIP & Dividend Re-Investment Plan (not used here — cash is recycled manually).\\
Blended Yield & Weighted average of all funds’ yields (\textasciitilde{}30–35 \%).\\
SPYG ↑ & Quarterly boost month (Mar, Jun, Sep, Dec).\\
Bill Cluster & Grouped payments by week (W1 Mortgage → W4 Subscriptions).\\
Net Cash Flow & Dividends − Bills − Margin Interest. Turns positive around month 9.\\
Income Phase & When dividends cover expenses and paycheck is fully invested.\\
Anchor Fund & Lower-yield fund that adds stability (SPYG in your case).\\
Rebalance Check & Quarterly review to keep each fund near 20 \%.\\
Self-Paying Margin & When dividends exceed interest and margin starts repaying itself.\\
\end{tabular}
\end{center}
\subsubsection{Section 3 – Quick-Start Checklist}
\label{sec:orgf7d275e}
\begin{itemize}
\item Direct your paycheck → savings.
\item Keep savings between \$890 and \$905.
\item Automate \$125 weekly (≈ \$500 monthly) to portfolio.
\item Route dividends back to savings.
\item Allow margin to auto-refill when below \$890.
\item Watch dividends cover more bills each quarter.
\item Increase contribution only after two months of stability.
\item Avoid margin over 25 \% of portfolio value.
\item Check fund weighting (20 \% each).
\item Stay the course — it snowballs quietly but powerfully.
\end{itemize}
\subsubsection{Closing Message}
\label{sec:org94e80a0}
\begin{quote}
``You’re not chasing the market — you’re building a machine. Each deposit, each dividend, and each margin cycle is one more turn of the wheel that eventually spins on its own.''
\end{quote}
\end{document}
